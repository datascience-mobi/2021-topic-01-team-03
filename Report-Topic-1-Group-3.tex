% Options for packages loaded elsewhere
\PassOptionsToPackage{unicode}{hyperref}
\PassOptionsToPackage{hyphens}{url}
%
\documentclass[
]{article}
\usepackage{lmodern}
\usepackage{amssymb,amsmath}
\usepackage{ifxetex,ifluatex}
\ifnum 0\ifxetex 1\fi\ifluatex 1\fi=0 % if pdftex
  \usepackage[T1]{fontenc}
  \usepackage[utf8]{inputenc}
  \usepackage{textcomp} % provide euro and other symbols
\else % if luatex or xetex
  \usepackage{unicode-math}
  \defaultfontfeatures{Scale=MatchLowercase}
  \defaultfontfeatures[\rmfamily]{Ligatures=TeX,Scale=1}
\fi
% Use upquote if available, for straight quotes in verbatim environments
\IfFileExists{upquote.sty}{\usepackage{upquote}}{}
\IfFileExists{microtype.sty}{% use microtype if available
  \usepackage[]{microtype}
  \UseMicrotypeSet[protrusion]{basicmath} % disable protrusion for tt fonts
}{}
\makeatletter
\@ifundefined{KOMAClassName}{% if non-KOMA class
  \IfFileExists{parskip.sty}{%
    \usepackage{parskip}
  }{% else
    \setlength{\parindent}{0pt}
    \setlength{\parskip}{6pt plus 2pt minus 1pt}}
}{% if KOMA class
  \KOMAoptions{parskip=half}}
\makeatother
\usepackage{xcolor}
\IfFileExists{xurl.sty}{\usepackage{xurl}}{} % add URL line breaks if available
\IfFileExists{bookmark.sty}{\usepackage{bookmark}}{\usepackage{hyperref}}
\hypersetup{
  pdftitle={Topic 1 Group 3: Drug viability screens for oncological and non-oncological treatments},
  pdfauthor={Cedrik Neber, Lea Ahrens, Lennard Kleemann, Ilya Schneider, Xenia Quaas},
  hidelinks,
  pdfcreator={LaTeX via pandoc}}
\urlstyle{same} % disable monospaced font for URLs
\usepackage[margin=1in]{geometry}
\usepackage{graphicx,grffile}
\makeatletter
\def\maxwidth{\ifdim\Gin@nat@width>\linewidth\linewidth\else\Gin@nat@width\fi}
\def\maxheight{\ifdim\Gin@nat@height>\textheight\textheight\else\Gin@nat@height\fi}
\makeatother
% Scale images if necessary, so that they will not overflow the page
% margins by default, and it is still possible to overwrite the defaults
% using explicit options in \includegraphics[width, height, ...]{}
\setkeys{Gin}{width=\maxwidth,height=\maxheight,keepaspectratio}
% Set default figure placement to htbp
\makeatletter
\def\fps@figure{htbp}
\makeatother
\setlength{\emergencystretch}{3em} % prevent overfull lines
\providecommand{\tightlist}{%
  \setlength{\itemsep}{0pt}\setlength{\parskip}{0pt}}
\setcounter{secnumdepth}{-\maxdimen} % remove section numbering

\title{Topic 1 Group 3: Drug viability screens for oncological and
non-oncological treatments}
\author{Cedrik Neber, Lea Ahrens, Lennard Kleemann, Ilya Schneider, Xenia Quaas}
\date{19 7 2021}

\begin{document}
\maketitle

Drug repurposing is a strategy in which already authorized drugs are
used to treat diseases for which they were not intended. This has
advantages, especially in terms of cost, time and the risk to fail in
research and development of a new drug. In this context, computational
approaches are gaining in importance, as they allow large amounts of
data to be analyzed.

According to the Global Cancer Statistics 2020, brain cancer is a rarer
cancer type, accounting for 2,5\% of all new cancers. Nevertheless, its
mortality rate is comparatively high, which makes drug repurposing an
interesting application.

\#Structure of our project

This project uses data sets generated by the Broad Institute using the
PRISM strategy. It investigates the general question of whether it is
possible to predict the effectiveness of a drug in the treatment of
brain cancer.

To explore our main question, we examined four milestones:

\emph{1) How can we distinguish the most effective drugs?} \emph{2) What
are the targets of the effective drugs?} \emph{3) Are there any genetic
markers that are specific for brain cancer subtypes?} \emph{4) What
other factors contribute to drug and effectiveness prediction? }

General structure: Main question: Is it possible to predict the
effectiveness of a drug? Further breakdown: 1) How can we distinguish
the most effective drugs? - Create separate data frames for each dose -
Define the threshold to select effective drugs - Find drugs that are
effective in mutiple doses

\begin{enumerate}
\def\labelenumi{\arabic{enumi})}
\setcounter{enumi}{1}
\tightlist
\item
  What are the targets of the effective drugs?
\item
  Are there any genetic markers that are specific for brain cancer
  subtypes?
\end{enumerate}

\begin{itemize}
\tightlist
\item
  Examine gene targets of the effective drugs
\item
  Find a relationship between gene targets and brain cancer subtypes
\item
  Wilcoxon Signed Rank Test
\end{itemize}

\begin{enumerate}
\def\labelenumi{\arabic{enumi})}
\setcounter{enumi}{3}
\tightlist
\item
  What other factors contribute to drug and effectiveness prediction?
\end{enumerate}

\begin{itemize}
\tightlist
\item
  Estimate the relevant variables
\item
  Look for correlations between variables and leave only variables with
  high variance
\item
  Reduce the amount of dimesions by conducting a PCA
\item
  Regression model with PCA
\item
  Compare the real effectiveness with the estimated one
\end{itemize}

Rough version with possible visualizations:

Q1: - Visualization of efficiencies for every doses via eight
\emph{histograms} with median - Visualization of effective drugs after
filtering for drugs effective in every cell type via Cedrik's
\emph{clustering} - Showing differences of efficiencies between subtypes
by \emph{boxplots}?

Q3: - Visualization of most relevant genes in cell lines via a
\emph{heatmap} with shows the degree of expression

\end{document}
